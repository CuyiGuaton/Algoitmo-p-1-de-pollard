\documentclass[12pt,letterpaper]{scrartcl}
\usepackage{lipsum}
\usepackage[utf8]{inputenc}
\usepackage{amsmath}
\usepackage{amsfonts}
\usepackage{amssymb}
\usepackage{graphicx}
\usepackage[left=3cm,right=2.5cm,top=2.5cm,bottom=2.5cm]{geometry}
\usepackage[]{algorithm2e}
\author{Don cuyi}

%Color
\usepackage{color}
\definecolor{nred}{RGB}{174,49,54}
\definecolor{nblue}{RGB}{86,99,146}
\definecolor{nalgo}{RGB}{188,139,76}
\usepackage{sectsty}
\sectionfont{\color{nred}}
\subsectionfont{\color{nblue}}
\subsubsectionfont{\color{nalgo}}

%Hipervinculos
\usepackage{hyperref}

\usepackage{fancyhdr}
\pagestyle{fancy}
\fancyhf{}
\fancyhead[L]{Algo. p-1 de Pollard}
\fancyhead[C]{Lic. en ciencia de la computación}
\fancyhead[R]{USACH}

%interlineado
\renewcommand{\baselinestretch}{1.2}

%\bibitem{Yahoo} \textsc{Andres G} (2009),
%\textbf{¿Generar números aleatorios negativos en Lenguaje C?} En \textsc{Yahoo! respuestas}
%Recuperado el el 23 del julio del 2014
%\url{https://es.answers.yahoo.com/question/index?qid=20091121055249AAUQH3N}

\newcommand{\biblio}[7]{
\bibitem{#1} \textsc{#2} (#3),
\textbf{#4} En \textsc{#5}
Recuperado el #6
\url{#7}
}

% Last, F. M. (Year Published) Book. City, State: Publisher.
\newcommand{\book}[5]{
\bibitem{#1} \textsc{#2} (#3),
\textbf{#4}  \textsc{#5} Estado: Publicado
}

\begin{document}

\begin{titlepage}

\begin{center}

{\Large { Licenciatura en ciencia de la computación} }

\includegraphics[scale=1]{UDSCNRJ}
\\[1cm]

{\Huge \textsc{Algoritmo p-1 de Pollard}}\\[0.7cm]

{\huge  Matemática Computacional}\\[2cm]


\begin{minipage}[l]{0.4\textwidth}
	\begin{flushleft}
	\linespread{1}
		\textbf{\textsf{Profesor:}}\\
		\large Nicolas Thériault
	\end{flushleft}
\end{minipage}
\begin{minipage}[l]{0.4\textwidth}

	\begin{flushright}

		\textbf{\textsf{Autor:}}\\
		\linespread{1}
		\large Sergio Salinas\\
		\large Danilo Abellá\\

	\end{flushright}
\end{minipage}

\end{center}

\end{titlepage}



\newpage

\tableofcontents

\newpage
\section{Introducción}

El informe trata sobre el analisis del los tiempos de ejecución del algoritmo p-1 de Pollard, para implementarlo se uso el lenguage C junto con la librería GMP en su versión 6.

\newpage
\section{Formulación experimentos}

Se probo el resultado con los tres números pedidos, pero para que el tiempo sea más exacto se ejecuto el algoritmo por durante un minuto y se calculo el primedio, los resultados son los siguientes.\\
	
\begin{center}
\begin{tabular}{|c|c|c|}
\hline 
n & B & Tiempo \\ 
\hline 
28742705413 & 9973 & 0.003851 \\ 
\hline 
45524252104894451218081 & 107 & 0.000052 \\ 
\hline 
17650684120269601571820630421347...
 & 655 & 0.009753 \\ 
\hline 
\end{tabular} 
\end{center}

Los datos dan indicios que el tiempo de ejecución depende más del valor que alcance B que del tamaño del número, para estar más seguros de esto se ejecuto el programa mil veces con valores al azar que van desde el 2 al 1000000000000 y se hicieron dos gráficos, uno que compara el tiempo vs B y otro que compara tiempo vs Número.\\

Para que el tiempo sea más preciso se repetitio cada ejecución por durante el lapso de un 1 segundo y se calculo un promedio entre todos los tiempos.\\

Se uso una función del gmp que calculaba la probabilidad de que un número sea primo, a la hora generar números al azar para crear la gráfica, se usaba esta función para saber si el número resultante era posiblemente primo, si lo era se descartaba y se probaba otro. Esta función no fue agregada al tiempo de ejecución del algoritmo.\\

La tabla resultante de puede ver en en informe/DATOS.txt, la primera columna es el número probado, la segunda el valor de B alcanzado y la tercera el tiempo de ejecución.\\
\newpage

\section{Información de Hardware y Software}

\subsection{ Notebook - Danilo Abellá}
\subsubsection{Software}
\begin{itemize}
\item SO: Xubuntu 16.04.1 LTS
\item GMP Library
\item Mousepad 0.4.0
\end{itemize}

\subsubsection{Hardware}
\begin{itemize}
\item AMD Turion(tm) X2 Dual-Core Mobile RM-72 2.10GHz
\item Memoria (RAM): 4,00 GB(3,75 GB utilizable)
\item Adaptador de pantalla: ATI Raedon HD 3200 Graphics
\end{itemize}



\subsection{Notebook - Sergio Salinas}
\subsubsection{Software}
\begin{itemize}
\item  SO: ubuntu Gnome 16.04 LTS
\item Compilador: gcc version 5.4.0 20160609 
\item Editor de text: Atom
\end{itemize}

\subsubsection{Hardware}
\begin{itemize}
\item Procesador: Intel Core i7-6500U CPU  2.50GHz x 4 
\item Video: Intel HD Graphics 520 (Skylake GT2) 
\end{itemize}

\newpage


\section{Curvas de desempeño de resultados}

\subsection{Número vs Tiempo}

\begin{center}
\includegraphics[scale=1]{NumvsTiempo.png} 
\end{center}

\newpage
\subsection{B vs Tiempo}

\begin{center}
\includegraphics[scale=1]{BvsTiempo.png} 
\end{center}
\newpage
\section{Conclusiones}
	
Con los primeros valores probados y en las gráficas queda claro que el tiempo se ejecución se mantiene constante independiente del tamaño del número, pero a medida que B va creciendo también lo va haciendo el tiempo, el caso más estremo fue el visto con el número 761126431349 que se demoro 0.087253s y alcanzo un valor B de 213623, mientras que el resto de los números que tenían valores B menores a 5 su tiempo de ejecución era de 0.000001s.


\end{document}
